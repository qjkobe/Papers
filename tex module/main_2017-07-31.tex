\documentclass[preprint,review,10pt]{elsarticle}

%% Use the options 1p,twocolumn; 3p; 3p,twocolumn; 5p; or 5p,twocolumn
%% for a journal layout:
%%\documentclass[final,1p,times]{elsarticle}
%%\documentclass[final,1p,times,twocolumn]{elsarticle}
%%\documentclass[final,3p,times]{elsarticle}
%\textwidth 17cm
%\textheight 24cm
%\hoffset -20mm
%\voffset -20mm
%\linespread{2.0}

%\def\ds{\displaystyle}%\displaystyle}
%\usepackage{cite}
\usepackage{hyperref}
\usepackage{amsmath,amssymb}
\usepackage{graphicx}
\usepackage{graphics}
\usepackage{amsmath, amssymb}

\usepackage {graphicx,fancyhdr}
\usepackage{graphics, color}
%%%%%%\usepackage{paisubfigure}
%\usepackage{amsthm}
\usepackage{flafter}
%\usepackage[subfigure]{caption2}
\numberwithin{equation}{section}
\usepackage{multirow}


%
\usepackage{amssymb}
\usepackage{subfig}


%%%%%%%%%%%%%%%%%%%%%%%%�Զ���%%%%%%%%%%%%%%%%%%%%%%%%%%%%%%%%%%%%%%%%%%%%%%%%%%%%%%%%%%%%

\def\d{\delta}
\def\ds{\displaystyle}
\def\e{{\epsilon}}
\def\eb{\bar{\eta}}
\def\enorm#1{\|#1\|_2}
\def\Fp{F^\prime}
\def\fishpack{{FISHPACK}}
\def\fortran{{FORTRAN}}
\def\gmres{{GMRES}}
\def\gmresm{{\rm GMRES($m$)}}

\def\norm#1{\|#1\|}
\def\wb{{\bar w}}
\def\zb{{\bar z}}



%%%%%%%%%%%%%%%%%%%%%%%%%%%%%%%%%%%%%%%%%%%%%%%%%%%%%%%%%%%
\newcommand\Kc{{\mathcal{K}}}
\newcommand\Bc{{\mathcal{B}}}
\newcommand\Cc{{\mathcal{C}}}
\newcommand\Ec{{\mathcal{E}}}
\newcommand\Tc{{\mathcal{T}}}
\newcommand{\pe}{\psi}
\def\d{\delta}
\def\D{\Delta}
\def\ds{\displaystyle}
\def\e{{\epsilon}}
\def\O{{\Omega}}
\def\o{{\omega}}
\def\n{\nabla}
\def\Eb{\bar{E}}
\def\enorm#1{\|#1\|_2}
\def\Fp{F^\prime}
\def\fishpack{{FISHPACK}}
\def\fortran{{FORTRAN}}
\def\gmres{{GMRES}}
\def\gmresm{{\rm GMRES($\Kc$)}}
\def\norm#1{\|#1\|}
\def\vn{\vec n}
\def\qb{\bar q}
\def\fb{{\bar f}}
\def\ub{{\bar u}}
\def\pb{{\bar p}}
\def\tb{{\bar t}}
\def\vb{\bar v}
\def\mb{\bar m}
\def\p{{\partial}}
\def\Kc{{\mathcal{K}}}
\def\Bc{{\mathcal{B}}}
\def\Cc{{\mathcal{C}}}
\def\Ec{{\cal E}}
\def\Mc{{\cal M}}
\def\no{{\nonumber}}
\def\a{{\alpha}}
\def\b{{\beta}}
\def\var{\varepsilon}
\def\g{{\gamma}}
\def\l{{\lambda}}
\def\G{{\Gamma}}
\def\k{{\kappa}}
\def\i{{\infty}}
\def\s{{\sigma}}
\def\r{{\rightarrow}}
\def\t{\tau}
\def\div{{\mbox{div}}}
\def\min{\mbox{min}}
%%%%%%%%%%%%%%%%%%%%%%%%%%%%%%%%%%%%%%%%%%%%%%%%%%%%%%%

% some definitions of bold math italics to make typing easier.
% They are used in the corollary.

\def\bfE{\mbox{\boldmath$E$}}
\def\bfG{\mbox{\boldmath$G$}}
%%%%%%%%%%%%%%%%%%%%%%%%%%%%%%%%%%%%%%%%%%%%%%%%%%%%%%%%%%%%%%%%%%%%%%%%%%%%%%%%%%%%%%%%%%





\begin{document}

\begin{frontmatter}
%% Title, authors and addresses

%% use the tnoteref command within \title for footnotes;
%% use the tnotetext command for theassociated footnote;
%% use the fnref command within \author or \address for footnotes;
%% use the fntext command for theassociated footnote;
%% use the corref command within \author for corresponding author footnotes;
%% use the cortext command for theassociated footnote;
%% use the ead command for the email address,
%% and the form \ead[url] for the home page:
%% \title{Title\tnoteref{label1}}
 %% \tnotetext[label1]{}
%% \author{Name\corref{cor1}\fnref{label2}}
%% \ead{email address}
%% \ead[url]{home page}
%% \fntext[label2]{}
%% \cortext[cor1]{}
%% \address{Address\fnref{label3}}
%% \fntext[label3]{}
\journal{Computers and Mathematics with Applications}
 \tnotetext[label1]{The project is supported by  NSF of China (11601315) and (11601241),
 the NUPTSF grant (NY215067), the Natural Science Foundation of Jiangsu Province grant (BK20160877) , and Shanghai Sailing Program (16YF1404000). }

\title{Local discontinuous Galerkin methods
with TVD Runge-Kutta time-marching for nonlinear Carburizing model\tnoteref{label1}}




%\author[lab1]{ Ying Li\corref{cor1},  }
%\cortext[cor1]{Corresponding author. E-mail address: yinglotus@t.shu.edu.cn}
%
%\author[lab2]{ 1}
%
%
%\address[lab1]{School of Computer Engineering and Science, Shanghai University, Shanghai, 2000444, China }
%\address[lab2]{College of Science, Shanghai University, Shanghai, 2000444, China }


\begin{abstract}


\end{abstract}


\begin{keyword}


\end{keyword}

\end{frontmatter}




\section{Introduction}%{INTRODUCTION}%%

\setcounter{equation}{0}

A fully discrete local discontinuous Galerkin (LDG) scheme
coupled with total variation diminishing Runge-Kutta time discretization,
for solving a nonlinear carburizing model
is introduced and analyzed in this paper.
The one-dimensional carburizing model is given by
\begin{equation} \label{eq:model}
\left\{ \begin{array}{ll}
c_t = (D(c)c_x)_x,  \\
c(x,0) = c_0(x),  \\
c(-L,t)=c_l, \quad c(L,t)=c_r,
                \end{array} \right.
\end{equation}
in $x \in \Omega= [-L,L]$ and $t\in [0,T]$.

In this paper, we would like to adopt the following hypothesis (\textbf{H}) for $D(c)$:
\begin{enumerate}
  \item There exist positive constants $d_{\star}$ and $d^{\star}$
  such that $d_{\star} \le D(c) \le d^{\star}$.
  \item $D(c)$ is uniformly Lipschitz continuous with respect to $c$, i.e, there exists
  positive constant $C_{\star}$ such that $|D(c_1)-D(c_2)|\le C_{\star}|c_1-c_2|$.
\end{enumerate}


\section{The LDG schemes}
\setcounter{equation}{0}
\label{sec2}
In this section, we will present the semi-discrete and fully-discrete LDG schemes
for solving the carburizing problem (\ref{eq:model}).


\section{Preliminaries}

\section{The }%{THE SECOND-ORDER STABILIZED FINITE ELEMENT METHOD}%%




\section{Numerical simulations}%{NUMERICAL SIMULATIONS}%




\subsection{Rates of convergence study}

\subsection{Rayleigh-Taylor instability}


\section{Conclusion}%{CONCLUSION}%
In this paper, we proposed a second-order mixed stabilized finite element method based on pressure projection method for variable density incompressible flows.
 The new method uses a second-order time splitting, solving
separately the transport equation  and the momentum equation.
To be more specific, the space discretization uses the same low order  $P_{1}$ finite element spaces over triangles.
 This method is designed by the difference of a consistent and under-integrated mass matrix of the pressure
 in order to stabilize the lowest equal-order finite element pairs in the last two equations.
  The time discretizations uses a second-order accurate scheme.
  The stability proof of the method we proposed for variable density
flows was given in the paper.




To verify the correctness of the method, it has been applied to the
test cases previously considered in the literature.
First, the rates of convergence of the method were given and
we obtained a better convergence rates compared with the results presented in time.
At the same time, the space rates of convergence of the method were proved to be
 in accordance with the theoretical expected ones,  leading so to an accurate solver.
Then, the simulation of the viscous Rayleigh-Taylor instability was also investigated.
The simulation results coincided with the law of physics and are very close to  the results presented in the literature.
Last, we considered the falling bubble test to investigate the robustness property of the scheme with regards to
high density ratios. We obtained very good results.
Compared with some established methods,
numerical results show that new method exhibited good stability
behavior even the high density ratios or the high Atwood number are used
in computation.





\begin{thebibliography}{10}

\bibitem{R1}   J. L. Guermond and  A. Salgado,   A splitting method for incompressible flows with variable density based on a pressure Poisson equation, J. Comput. Phys.  228(2009)2834-2846.
\bibitem{R2}   J. H. Pyo and J. Shen,   Gauge-Uzawa methods for incompressible flows with variable density,  J. Comput. Phys. 221(2007)181-197.
\bibitem{R3}   A. J. Chorin,   Numerical solution of the Navier-Stokes equations, Math. Comput.  22(1968)745-762.
\bibitem{R4}   A. J. Chorin,  On the convergence of discrete approximations to the Navier-Stokes equations, Math. Comput.  23(1969)341-353.
\bibitem{R5}   R. Temam,    Navier-Stokes Equations, Studies in Mathematics and Its Applications 2, North-Holland, Amsterdam, 1977.
\bibitem{R6}   R. Temam,    Sur l'approximation des ��quatious de Navier-Stokes par la m��thode des pas fractionnaires (II), Arch. Ration. Mech. An.  33(1969)377-385.
\bibitem{R7}   J. L. Guermond, P. Minev and J. Shen,   An overview of projection methods for incompressible flows, Methods. Appl. Mech. Eng.  195(2006)6011-6045.
\bibitem{R8}   J. L. Guermond and L. Quartapelle,    A projection FEM for variable density incompressible flows, J. Comput. Phys.  165( 2000)167-188.
\bibitem{R9}   Y. Fraigneau, J. L. Guermond and L. Quartapelle,  Approximation of variable density incompressible flows by means of finite elements and finite volumes,  Commun. Numer. Meth. En. 17(2001)893-902.
\bibitem{R10}  G. P. Puckett, A. S. Almgren, J. B. Bell, D. L. Marcus and W. Rider,    A high-order projection method for tracking fluid interfaces in variable density incompressible flows, J. Comput. Phys.   130(1997) 269-282.
\bibitem{R11}  G. C. Buscaglia and R. Codina,    Fourier analysis of an equal-order incompressible flow solver stabilized by pressure gradient projection, Int. J. Numer. Methods. Fluids. 34(2000)65-92.
\bibitem{R12}  J. B. Bell and D. L. Marcus,    A second order projection method for variable-density flows, J. Comput. Phys.  101(1992)334-348.
\bibitem{R13}  A. S. Almgren , J. B. Bell,  P. Colella, L.H. Howell and M.L. Welcome,   A conservative adaptive projection method for the variable density incompressible Navier-Stokes equations, J. Comput. Phys.  142(1998) 1-46.
\bibitem{R14}  C. Calgaro, E. Creuse and T. Goudon,     An hybrid finite volume-finite element method for variable density incompressible flows,  J. Comput. Phys.  227(2008)4671-4696.
\bibitem{R15}  J. L. Guermond and A. J. Salgado,   Error analysis of a fractional time-stepping technique for incompressible flows with variable density, SIAM J. Numer. Anal. 49(2011)917-944.
\bibitem{R16}  J. Li and Y. He,   A stabilized finite element method based on two local Gauss integrations for the Stokes equations, J. Comput. Appl. Math. 214(2008)58-65.
\bibitem{R17}  Y. He and J. Li,    A stabilized finite element method based on local polynomial pressure projection for the stationary Navier-Stokes equations, Appl. Numer. Math.  58(2008)1503-1514.
\bibitem{lhc} J. Li, Y. He and Z. Chen,  A new stabilized finite element method  for the transient Navier-Stokes equations, Comput. Methods. Appl. Mech. Eng. 197(2007)22-35.
\bibitem{R18}  J. L. Guermond,    Stabilization of Galerkin approximations of transport equations by subgrid modeling,  M2AN Math. Model. Numer. Anal.  33(1999)1293-1316.
\bibitem{Donea} J. Donea, A. Huerta, Finite Element Methods for Flow Problems, John Wiley \& Sons Ltd, England, 2004.
\bibitem{R19}  P. B. Bochev, C. R. Dohrmann and M. D. Gunzburger,   Stabilization of low-order mixed finite elements for the stokes equations, SIAM J. Numer. Anal.   44(2006)82-101.
\bibitem{R20}  C. R. Dohrmann and P. B. Bochev,   A stabilized finite element method for the Stokes problem based on polynomial pressure projections,  Int. J. Numer. Methods. Fluids.  46(2004)183-201.
\bibitem{R21}  V. Girault and P. A. Raviart,   Finite Element Methods for Navier-Stokes Equations, Springer Series in Computational Mathematics,  Springer-Verlag, Berlin, 1986.
\bibitem{R22}  A. Ern and J. L. Guermond,   Theory and Practice of Finite Elements, Applied Mathematical Sciences,
  Springer-Verlag, New York, 2004.
%\bibitem{R23}  R. Temam,    Navier-Stokes Equations, Theory and Numerical Analysis, Reprint of the 1984 edition, AMS Chelsea Publishing, Providence, RI, 2001.
\bibitem{R24}  R. A. Adams,   Sobolev Spaces, Academic Press, New York, 1975.
\bibitem{R25}  L. C. Evans,   Partial Differential Equations, American Mathematical Society, 1998.
\bibitem{R26}  Z. Chen,   Finite Element Methods and Their Applications, Scientific Computation, Springer, Berlin, 2005.
\bibitem{R27}  G. Tryggvason,   Numerical simulations of the Rayleigh-Taylor instability,  J. Comput. Phys.   75(1988)235-282.
\bibitem{R28}  T. Schneider, N. Botta, K. J. Geratz and R. Klein,    Extension of finite volume compressible flow solvers to multidimensional, variable density zero Mach number flows,  J. Comput. Phys. 155(1999)248-286.
\bibitem{R29}  U. Rasthofer, F. Henke, W. A. Wall and V. Gravemeier,   An extended residual-based variational multiscale method for two-phase flow including surface tension, Comput. Methods. Appl. Mech. Eng.   200(2011) 1866-1876.
\bibitem{R30}  T. P. Fries,   The intrinsic XFEM for two-phase flows, Int. J. Numer. Methods. Fluids.   60(2009)437-471.
%
%
%\begin{thebibliography}{}
%\bibitem{1} J.L. Guermond, A. Salgado, A splitting method for incompressible flows with variable density based on a pressure Poisson equation, Journal of Computational Physics 228(2009) 2834-2846.
%\bibitem{2} J.H. Pyo, J. Shen, Gauge-Uzawa methods for incompressible flows with variable density, Journal of Computational Physics 221(2007) 181-197.
%\bibitem{3} J.L. Guermond, L. Quartapelle,  A projection FEM for variable density incompressible flows, Journal of Computational Physics 165(2000) 167-188.
%\bibitem{4} P.L. Lions, Mathematical Topics in Fluid Mechanics, Clarendon, Oxford, 1996.
%\bibitem{5} C. Liu, N.J. Walkington, Convergence of numerical approximations of the incompressible Navier-Stokes equations with variable density and viscosity, SIAM Journal on Numerical Analysis 45(2007) 1287-1304.
%\bibitem{6} J.L. Guermond, P. Minev , J. Shen,  An overview of projection methods for incompressible flows, Computer Methods in Applied Mechanics and Engineering 195(2006) 6011-6045.
%\bibitem{7} J.L. Guermond, A. Salgado,  A fractional step method based on a pressure Poisson equation for incompressible flows with variable density, Numerical Analysis (2008) 913-918.
%\bibitem{8} J.L. Guermond, A. Salgado, Error analysis of a fractional time-stepping technique for incompressible flows with variable density, SIAM Journal on Numerical Analysis 49(2011) 917-940.
%\bibitem{9} A.J. Chorin, Numeriacl solution of the Navier-Stokes equations, Mathematics of Computation 22(1968) 745-762.
%\bibitem{10} A.J. Chorin, On the convergence of discrete approximations to the Navier-Stokes equations, Mathematics of Computation  23(1969) 341-353.
%\bibitem{11} R. Temam, Navier-Stokes Equations, Studies in Mathematics and Its Applications 2, North-Holland, Amsterdam, 1977.
%\bibitem{12} R. Temam, Sur l'approximation des ��quatious de Navier-Stokes par la m��thode des pas fractionnaires (II)[J], Archive for Rational Mechanics and Analysis 33(1969) 377-385.
%\bibitem{13} Y. Fraigneau, J.L. Guermond, L. Quartapelle,  Approximation of variable density incompressible flows by means of finite elements and finite volumes, Communications in Numerical Methods in Engineering 17(2001) 893-902.
%\bibitem{14} G.P. Puckett, A.S. Almgren, J.B. Bell, D.L. Marcus, W. Rider,  A high-order projection method for tracking fluid interfaces in variable density incompressible flows, Journal of Computational Physics 130(1997)  269-282.
%\bibitem{15} G.C. Buscaglia, R. Codina,  Fourier analysis of an equal-order incompressible flow solver stabilized by pressure gradient projection, International
%Journal for Numerical Methods in Fluids 34(2000) 65-92.
%\bibitem{16} J.B. Bell, D.L. Marcus,  A second order projection method for variable-density flows, Journal of Computational Physics 1992; 101:334-348.
%\bibitem{17} A.S. Almgren, J.B. Bell, P. Colella, L.H. Howell, M.L. Welcome, A conservative adaptive projection method for the variable density incompressible Navier-Stokes equations, Journal of Computational Physics 142(1998) 1-46.
%\bibitem{18} C. Calgaro, E. Creuse, T. Goudon,  An hybrid finite volume-finite element method for variable density incompressible flows, Journal of Computational Physics 227(2008) 4671-4696.
%\bibitem{19} Y. He, J. Li,  Convergence of three iterative methods based on the finite element discretization for the stationary Navier-Stokes equations, Computer Methods in Applied Mechanics and Engineering 198(2009) 1351-1359.
%\bibitem{20} R. Codina, J. Blasco, G. Buscaglia, A. Huerta,  Implementation of a stabilized finite element formulation for the incompressible Navier-Stokes equations based on a pressure gradient projection, International Journal for Numerical Methods in Fluids 37(2001) 419-444.
%\bibitem{21} C. Dohrmann, P. Bochev,  A stabilized finite element method for the Stokes problem based on polynomial pressure projections, International Journal for Numerical Methods in Fluids   46(2004) 183-201.
%\bibitem{22} L. Cattabriga,  Su un problema al contorno relativo al sistema di equazioni di Stokes, Rendiconti del Seminario Matematico della Universit�� di Padova 31(1961) 308-340.
%\bibitem{23} R. Temam,  Navier-Stokes Equations: Theory and Numerical Analysis, North-Holland, Amsterdam, 1984.
%\bibitem{24} V. Girault, P.A. Raviart,  Finite Element Method for Navier-Stokes Equations: Theory and Algorithms, Springer-Verlag, Berlin, Heidelberg, 1986.
%%\bibitem{25} R.A. Adams,  Sobolev Spaces, Academic Press, New York, 1975.
%%\bibitem{26} L.C. Evans,  Partial Differential Equations, American Mathematical Society,1998.
%\bibitem{27} A. Ern, J.L. Guermond, Theory and Practice of Finite Elements, Springer-Verlag, New York, 2004.
%\bibitem{28} Z. Chen, Finite Element Methods and Their Applications, Scientific Computation, Springer, Berlin, 2005.
%\bibitem{29} G. Tryggvason, Numerical simulations of the Rayleigh-Taylor instability, Journal of Computational Physics 75(1988)  235-282.
%\bibitem{30} T. Schneider, N. Botta, K.J. Geratz, R. Klein,  Extension of finite volume compressible flow solvers to multidimensional, variable
%density zero Mach number flows, Journal of Computational Physics 155(1999)  248-286.
%\bibitem{31} U. Rasthofer, F. Henke, W.A. Wall, V. Gravemeier, An extended residual-based variational multiscale method for two-phase flow including surface tension, Computer Methods in Applied Mechanics and Engineering  200(2011)  1866-1876.
%\bibitem{32} T. P. Fries, The intrinsic XFEM for two-phase flows, International Journal for Numerical Methods in Fluids, 60(2009)  437-471.
\end{thebibliography}



\end{document}
\endinput
%%
%% End of file `elsarticle-regurization.tex'.
